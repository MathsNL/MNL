\documentclass{article}

\begin{document}
	\title{Mathematics Notation Language}
	
	\maketitle
	
	\section{Aims}
		\begin{description}
			\item[Minimising verbosity] Some notation for mathematics can be 
				verbose compared to others (compare, for example, the simple 
				notation for indices, yet the need for an entire $\log$ 
				function for a relatively similar operation).  We aim to 
				minimise this verbosity (of course, not all notation can be 
				minimal) and, through this, allow quick and easy writing and 
				typing of mathematics.  
			\item[Removing ambiguity] The standard equals operator ($=$) can 
				be an equivalence, an identity, a relation, or an assignment. 
				Whilst these operations are similar, they can be very different 
				and using the same operator is ambiguous. We aim to remove 
				this ambiguity, and make different operations clearly 
				different.  
			\item[Improving clarity] Along a similar vein, we aim to improve 
				notation such that similar operations appear similar (but not 
				the same). This is necessary such that those learning 
				mathematics can intuitively see that the operations are 
				similar. A perfect example is that of logs, indices and 
				roots. These operations are all very similar, yet the 
				notation is hugely different.  
			\item[Written--Typed conversion] As both programmers and 
				mathematicians, we aim to make conversion between written 
				and typed mathematics simple. Whether this be from written 
				notation to \TeX, or written notation to standard typed text.
		\end{description}
\end{document}
